\begin{enumerate}[label=\thesubsection.\arabic*.,ref=\thesubsection.\theenumi]
\numberwithin{equation}{enumi}

\item
A dc amplifier has an open loop gain of 1000 and two poles, a dominant one at 1kHz and a high frequency one whose location can be controlled. It is required to connect this amplifier in a negative feedback loop that provides a dc closed loop gain of 10 and a maximally flat response.Find the required value of H.
\\
\solution Given, open loop gain G is dependent on frequency and has two poles.Therefore,G(s) can be written as..,
\begin{align}
  G(s) &= \frac{G_0}{(1-\frac{s}{p_{1}})(1-\frac{s}{p_{2}})} \label{eq:open_loop_gain} 
\end{align}
where, p1 and p2 are poles of G(s).
Parameters given are shown in Table.\ref{table:Input_Table}:1
\begin{table}[!ht]
\centering
\input{./tables/ee18btech11005/ee18btech11005_1.tex}
\caption{1}
\label{table:Input_Table}
\end{table}
\begin{align}
    G_0 &= 1000\\
\text{Therefore.,} G(s)&= \frac{1000}{(1-\frac{s}{p_{1}})(1-\frac{s}{p_{2}})}  
\end{align}
Let, p1 be the dominant pole.
\begin{align}
    p_1 &= 2\pi10^3 rad/sec
\end{align}
Now, we connect the system in a negative feedback of feedback factor H.
We know that the closed loop gain of a negative feedback system is.,
\begin{align}
    T(s) &= \frac{G(s)}{1+G(s)H} \label{eq:transfer_function}
\end{align}
But,also given DC closed loop gain is 10. DC closed loop gain is given in equation.\ref{eq:dc_gain} 
\begin{align}
T(0) &= \frac{G_0}{1+G_0H} \label{eq:dc_gain} \\
\text{given.,}T(0)&=10\\
\text{and.,}G_0 &= 1000\\
\frac{1000}{1+1000H} &= 10\\
1+1000H &= 100\\
\implies H &= \frac{99}{1000} = 0.099 \label{eq:h_value}
\end{align}
from equation:\ref{eq:h_value}, the value of H is 0.099.Therefore, the feedback factor is 0.099.
%------------------------------
\item Find the location at which second ple can be placed.\\
\solution Given the negative feedback system should have maximally flat response.from equation.\ref{eq:transfer_function} and equation.\ref{eq:open_loop_gain} the transfer function is,
\begin{align}
    T(s) &= \frac{\frac{G_0}{(1-\frac{s}{p_{1}})(1-\frac{s}{p_{2}})}}{1+\frac{HG_0}{(1-\frac{s}{p_{1}})(1-\frac{s}{p_{2}})}}\\
    T(s) &= \frac{\frac{p_1p_2G_0}{(p_1-s)(p_2-s)}}{1+\frac{p_1p_2HG_0}{(p_1-s)(p_2-s)}}
\end{align}
\begin{align}
    T(s) &= \frac{p_1p_2G_0}{(p_1-s)(p_2-s) + p_1p_2HG_0}\\
    T(s) &= \frac{p_1p_2G_0}{p_1p_2-(p_1+p_2)s+s^2 + p_1p_2HG_0}\\
    T(s) &= \frac{p_1p_2G_0}{s^2-(p_1+p_2)s+(HG_0+1)p_1p_2} \label{eq:closed_loop}
\end{align}
The characteristics equation of above transfer function is.,
\begin{align}
    C.E &= s^2-(p_1+p_2)s+(HG_0+1)p_1p_2 \label{eq:ce}
\end{align}
In general, For a second order amplifier the C.E is.,
\begin{align}
    C.E &= s^2+2\zeta\omega_ns+\omega_n^{2} \label{eq:second_order_ce}
\end{align}
The quality factor Q of equation.\ref{eq:second_order_ce}, is given by
\begin{align}
    Q &= \frac{1}{2\zeta}
\end{align}
From equation.\ref{eq:ce},
\begin{align}
    \omega_n &= \pm{\sqrt{(HG_0+1)p_1p_2}}\\
    \zeta &= \pm{\frac{2p_1p_2}{\sqrt{(HG_0+1)p_1p_2}}}
\end{align}
Therefore,For the given second order amplifier with characteristic equation.\ref{eq:ce},the Q factor is,
\begin{align}
    Q &= \pm{\frac{\sqrt{(1+HG_0)p_1p_2}}{p_1+p_2}} \label{eq:Q}
\end{align}
For a negative feedback amplifier to achieve maximally flat response,
\begin{align}
    Q &= 0.7071\\
\end{align}
Therefore, substituting the values of Q and other parameters in equation.\ref{eq:Q},
\begin{align}
    0.7071 &= \pm{\frac{\sqrt{(1+0.099(1000))(2\pi10^3)p_2}}{2\pi10^3+p_2}}
\end{align}
Squaring on both sides and rearranging.,
\begin{align}
    (1+1000(0.099))2\pi10^3p_2 &= 0.7071^2{(2\pi10^3+p_2)}^2
\end{align}
\begin{align}
    (100)2\pi10^3 = 0.7071^2(2\pi10^3+p_2)^2\\
    \implies 0.5p_2^2-622037.2p_2+19733247.6=0
\end{align}
Solving above equation.,
\begin{align}
p_2 &= 31.7244 \text{ rad/sec}\\
p_2 &= 1244042.676 \text{ rad/sec}
\end{align}
But, Since p1 is dominating pole,p1 should be close to origin.
\begin{align}
    p_1 <<< p_2
\end{align}
\begin{align}
p_2 &= 1.244\text{ Mrad/sec}\\
\\
p_2 &= \frac{1.244M}{2\pi} \text{Hz}= 192.989\text{ kHz}\\
\end{align}
%---------------------
\item Verify roots of above equation using python code.
\begin{lstlisting}
codes/ee18btech11005/ee18btech11005_1.py
\end{lstlisting}
%-----------------
\item Now, Find the open loop transfer function and closed loop transfer function of the system.
\solution Substituting the value of p2 in  the equation \ref{eq:open_loop_gain} and \ref{eq:closed_loop}
\begin{align}
     G(s) &= \frac{1000}{(1-\frac{s}{10^3})(1-\frac{s}{192.989x10^3})}
\end{align}
\begin{align}
    T(s) &= \frac{192.989x10^9}{s^2-193x10^3s+192.989x10^{10}}
\end{align}
%--------------------------
\item Verify from the Bode plot of above closed loop transfer function that it has maximally flat response.\\
\solution The following code generates the bode plot of the transfer function.
\begin{lstlisting}
codes/ee18btech11005/ee18btech11005_2.py
\end{lstlisting}
\begin{figure}[!ht]
\centering
\includegraphics[width=\columnwidth]{./figs/ee18btech11005/ee18btech11005_1.eps}
\caption{1}
\label{fig:ee18btech11005_1}
\end{figure}
From the magnitude bode plot in figure.\ref{fig:ee18btech11005_1}:1 we can tell that the open loop transfer function after having negative feedback has maximally flat response for the obtained value of p2. 
\item Design a circuit that represents the above transfer function.\\
\solution The circuit can be designed using an operational amplifier having negative feedback.Consider the circuit shown in figure.\ref{fig:equivalent_control_system}
\begin{figure}[!ht]
	\begin{center}
			\resizebox{\columnwidth}{!}{\tikzstyle{block} = [draw, fill=blue!20, rectangle, 
    minimum height=3em, minimum width=6em]
\tikzstyle{sum} = [draw, fill=blue!20, circle, node distance=1cm]
\tikzstyle{input} = [coordinate]
\tikzstyle{output} = [coordinate]
\tikzstyle{pinstyle} = [pin edge={to-,thin,black}]

\begin{tikzpicture}[auto, node distance=2cm,>=latex']
    \node [input, name=input] {$V_s$};
    \node [sum, right of=input] (sum) {};
    \node [block, right of=sum] (controller) {$G$};
    \node [output, right of=controller] (output) {};
    \node [block, below of=controller] (feedback) {$H$};
    \draw [draw,->] (input) -- node {$V_s$} (sum);
    \draw [->] (sum) -- node {$V_i$} (controller);
    \draw [->] (controller) -- node [name=y] {$V_o$}(output);
    \draw [->] (y) |- (feedback);
    \draw [->] (feedback) -| node[pos=0.99]{$-$}  node [near end] {$V_f$} (sum);
\end{tikzpicture}
}
	\end{center}
\caption{}
\label{fig:equivalent_control_system}
\end{figure}
For a differential amplifier,
\begin{align}
  V_o &= G(s)(V_{in} - V) \\
  \text{where.,} V &= \frac{V_oR}{kR}=\frac{V_o}{k}\\
  V_o &= G(s)(V_{in} - \frac{V_o}{k})\\
  V_o &= G(s)V_{in} - \frac{G(s)V_o}{k}\\
  V_o &= \frac{G(s)V_{in}}{1+\frac{G(s)}{k}}
  \end{align}
  \begin{align}
  T(s) &= \frac{G(s)}{1+\frac{G(s)}{k}}\label{eq:Opamp_equivalent}
\end{align}
Where.., G(s) is open loop gain that varies with frequency.
Comparing equation.\ref{eq:transfer_function} and equation.\ref{eq:Opamp_equivalent}
We can design the equivalent circuit for the given problem.
\begin{align}
H &= \frac{1}{k}\\
\implies k &= \frac{1}{H}\\
\end{align}
We know.,H = 0.099
\begin{align}
k &= \frac{1}{0.099} = \frac{1000}{99}\\
k &= 10.101
\end{align}
Choose..,
\begin{align}
    R = 1000 \Omega
\end{align}
So, the final circuit is shown in figure.\ref{fig:final_circuit}.
\begin{figure}[!ht]
	\begin{center}
			\resizebox{\columnwidth}{!}{ \begin{circuitikz}
\ctikzset{bipoles/length=1cm}

\draw 
(0, 0) node[op amp] (opamp) {}
(opamp.-) -- (-1,0.35) -- (-1, 2)--(1.5,2) to[R=$9101\Omega$,*-*] (1.5,0){}
(opamp.out)--(1.5,0){}
(opamp.+) -- (-2,-0.35) node at(-2,-0.5) {$V_{in}$}
(1.5,2) to[R=$1000\Omega$] (1.5,4)--(2,4) node[ground]{}
(1.5,0) -- (2,0)
node at(2,-0.3){$V_o$}
node at(1.7,2){V}
(opamp.center) node[]{{$G(s)$}}
;\end{circuitikz}
}
	\end{center}
\caption{}
\label{fig:final_circuit}
\end{figure}
The above circuit ensures the closed loop transfer function has maximally flat response.
\end{enumerate}

