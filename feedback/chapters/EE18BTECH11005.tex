\begin{enumerate}[label=\thesubsection.\arabic*.,ref=\thesubsection.\theenumi]
\numberwithin{equation}{enumi}

\item
The non-inverting op-amp configuration shown in fig.\ref{fig:original_circuit} provides direct implementation of feedback loop.Assuming operational amplifier has infinite input resistance and zero output resitance.Find the expression for feedback factor.
\begin{figure}[!ht]
	\begin{center}
		
		\resizebox{\columnwidth}{!}{\begin{circuitikz}
\ctikzset{bipoles/length=1cm}

\draw 
(0, 0) node[op amp] (opamp) {}
(opamp.-) to[R,l_=$R_1$,*-*] (-2, 0.35) to (-2.5, 0.35) to (-2.5, 0.35) node[ground]{}
(opamp.-) --(-0.9,1) to[R=$R_2$] (1,1) -- (1,0) --(2,0) node at(2.3,0){$V_0$}
(opamp.out) to (1.5,0)--(1.5,-0.5) to[R=$R_L$] (1.5,-1.5) to (1.5,-1.5) node[ground]{}
(opamp.+) -- (-0.6,-0.35) to[R =$R_s$,*-*] (-2.6,-0.35) to[V=$V_s$] (-2.6,-2.4) node[ground]{}
;\end{circuitikz}

}
	\end{center}
\caption{}
\label{fig:original_circuit}
\end{figure}
\solution Let the gain of the operational amplifier be A.
The equivalent circuit of the amplifier is in fig.\ref{fig:equivalent_circuit}
\begin{figure}[!ht]
	\begin{center}
		
		\resizebox{\columnwidth}{!}{\usetikzlibrary{decorations.markings}
\begin{circuitikz}
\ctikzset{bipoles/length=1cm}

\draw 
(0, 0) to[V=$V_s$] (0,-1.5) to (0,-1.5) node[ground]{}
(0,0) -- (0,1)--(0.25,1) to[R=$R_s$] (1.5,1)  node at(1.8,1){$+$}
%(1.5,3) node[pos=10]{$V_i$}
(1.5,-1.25)  node at(1.7,-1.25){$-$} 
(1.5,-1.25) -- (1,-1.25) -- (1,-1.75) to[R=$R_1$] (1,-2.75) --(1,-3) node[ground]{}
(1,-1.5) to[R=$R_2$] (5,-1.5){}
(5,-1.5) -- (5,1) --(3.5,1) to[V=$GV_i$] (3.5,-0.5) node[ground]{}
(5,1) --(6,1) to[R=$R_l$,*-*] (6,-0.5) node[ground]{}
(6,1) --(6.5,1) node at(6.8,1){$V_0$}
node at(1.8,-0.3) {$V_i$}
node at(0.6,-1.75){$+$}
node at(0.6,-3){$-$}
node at(0.6,-2.5){$V_f$}
;\end{circuitikz}
}
	\end{center}
\caption{}
\label{fig:equivalent_circuit}
\end{figure}
From the equivalent circuit,
Applying Ohms law,
\begin{align}
V_0 &= A(V_+ - V_-) \label{eq:opamp_output}
\end{align}
Now,Applying voltage dividing rule
\begin{align}
V_- &= \sbrak{\frac{R_1}{R_1+R_2}}V_0
\end{align}
Substituting in equ.\ref{eq:opamp_output}
\begin{align}
    V_0 &= A(V_+-\sbrak{\frac{R_1}{R_1+R_2}}V_0)
    \\
\implies V_0 &= AV_+-A\sbrak{\frac{R_1}{R_1+R_2}}V_0
    \\
A(V_+)&=V_0+A\sbrak{\frac{R_1}{R_1+R_2}}V_0
\end{align}
But,
\begin{align}
    V_s &= V_+
\end{align}
because, no current flows through resistor,Rs Since,input resistance is given infinite in fig.\ref{fig:equivalent_circuit} 
The equation can be written as...,
\begin{align}
    V_0 &= A\sbrak{\frac{1}{1+\frac{AR_1}{R_1+R_2}}}V_s
    \\
 \text{Gain =}\frac{V_0}{V_s}&=\sbrak{\frac{A}{1+\frac{AR_1}{R_1+R_2}}}\label{eq:opamp_gain}
\end{align}
For a negative feedback system,
\begin{align}
   \frac{V_0}{V_i} &= \frac{A}{1+A\beta}
\end{align}
The equation.\ref{eq:opamp_output} looks exactly similar to the Gain of a negative feedback system with
\begin{itemize}
    \item Open loop gain = A
    \item Loop gain = P 
    \item Amount of feedback = F
    \item Feedback factor = f
\end{itemize}
where
\begin{align}
    A &= A\\
    P &= A\beta = A\frac{R_1}{R_1+R_2}\\
    F &= 1+A\beta = 1 + \frac{AR_1}{R_1+R_2}\\
    f &= \beta = \frac{R_1}{R_1+R_2}
\end{align}
Therefore,This operational amplifier can be modelled as a negative feedback system shown in the fig.\ref{fig:equivalent_control_system}
\begin{figure}[!ht]
	\begin{center}
			\resizebox{\columnwidth}{!}{\tikzstyle{block} = [draw, fill=blue!20, rectangle, 
    minimum height=3em, minimum width=6em]
\tikzstyle{sum} = [draw, fill=blue!20, circle, node distance=1cm]
\tikzstyle{input} = [coordinate]
\tikzstyle{output} = [coordinate]
\tikzstyle{pinstyle} = [pin edge={to-,thin,black}]

\begin{tikzpicture}[auto, node distance=2cm,>=latex']
    \node [input, name=input] {$V_s$};
    \node [sum, right of=input] (sum) {};
    \node [block, right of=sum] (controller) {$G$};
    \node [output, right of=controller] (output) {};
    \node [block, below of=controller] (feedback) {$H$};
    \draw [draw,->] (input) -- node {$V_s$} (sum);
    \draw [->] (sum) -- node {$V_i$} (controller);
    \draw [->] (controller) -- node [name=y] {$V_o$}(output);
    \draw [->] (y) |- (feedback);
    \draw [->] (feedback) -| node[pos=0.99]{$-$}  node [near end] {$V_f$} (sum);
\end{tikzpicture}
}
	\end{center}
\caption{}
\label{fig:equivalent_control_system}
\end{figure}
So, the feedback factor f..,
\begin{align}
     f &=\beta = \frac{R_1}{R_1+R_2}
\end{align}
\item Find the condition under which closed loop gain Af is almost entirely determined by the feedback network.
\solution For Af to entirely dependent on feedback network, it should be independent on A(open loop gain)
Af is given by..,
\begin{align}
    A_f &= \frac{A}{1+A\beta} \\
\end{align}
For Af to be independent on A..,
\begin{align}
 A\beta >> 1 \\
 A\frac{R_1}{R_1+R_2} >> 1 \\
 A >> 1 + \frac{R_2}{R_1} 
\end{align}
Under such condition..,
\begin{align}
    A_f &= \frac{1}{\beta} \\
    A_f &= \frac{R_1+R_2}{R_1}\\
    A_f &= 1+\frac{R_2}{R_1}
\end{align}
so, the necessary condition for Af depend only on feedback network is
\begin{align}
    A >> A_f
\end{align}
\item If the open loop voltage gain
\begin{align} 
A & = 10^4
\end{align}
Find the ratio of R2 and R1 to obtain a closed loop gain of 10.
\solution The closed loop gain gain Af is given by
\begin{align}
    A_f &= \frac{A}{1+A\beta}
        = \frac{A}{1+\frac{AR_1}{R_1+R_2}} = 10\\
    \text{where..,} A &= 10^4 \\
    10 &= \frac{10^4}{1+\frac{10^4}{1+\frac{R_2}{R_1}}}\\
\implies 1+\frac{R_2}{R_1} &= \frac{10^4}{\frac{10^4}{10}-1}
\\
1+\frac{R_2}{R_1} &= 10.010
\\
\frac{R_2}{R_1} &= 9.010
\end{align}
\item What is the amount of feedback in decibels?
\solution The value of F in decibals is given by 
\begin{align}
    F(dB) &= 20log(F)\\
\text{where..,} F &= 1+A\beta \\
F &= \frac{A}{A_f}\\
\text{where..,} A&=10^4 \\ A_f &= 10\\
F(dB) &= 20log(\frac{10^4}{10})=20log(1000)\\
F(dB) &= 60 dB
\end{align}
\item If A decreases by 20\%,what is the corresponding decrease in Af?
\solution Given
\begin{align}
A = 10^4
\end{align}
If A decrease by 20\% then,
the value of A is..,
\begin{align}
    A &= (1-0.2)10^4 \\
      &= 8000
\end{align}
For this value of A and ,
\begin{align}
    \frac{R_2}{R_1} = 9.010
\end{align}
The value of Af can be solved as follows,
\begin{align}
 A_f &= \frac{A}{1+\frac{A}{1+\frac{R_2}{R_1}}}\\
 A_f &= \frac{8000}{1+\frac{8000}{1+0.9010}}\\
 A_f &= 9.99749
\end{align}
The percentage change in Af is..,
\begin{align}
    fractionalchange &= \frac{10-9.99749}{10}\\
      &= 2.51x10^{-4}\\
     \% change in A_f &= 0.00251
\end{align}
Therefore Af decreases by 0.0025\% when A decreases by 20\%
\end{enumerate}
